%% For Use With NIH Biosketch LaTeX Class
%% Added to RMarkdown by Tyson S. Barrett, PhD

% Document class and options
\documentclass{nihbiosketch}
\usepackage{natbib}
\usepackage{bibentry}
\bibliographystyle{apalike}
\providecommand{\tightlist}{\setlength{\itemsep}{0pt}\setlength{\parskip}{0pt}}

% Pandoc header




\name{Nicholas Giangreco}
\eracommons{14654270}
\position{Quantitative Translational Scientist}


\begin{document}



  \nobibliography{yourbibfile.bib}


  \begin{education}
      University of Rochester & BS & 5/2014 & Biochemistry \\
      Columbia University & PhD & 11/2021 & Cellular, Molecular, and
Biomedical Studies \\
    \end{education}



\section*{Personal Statement}\label{personal-statement}
\addcontentsline{toc}{section}{Personal Statement}

I have led and contributed to precision medicine research as a PhD
trainee and currently in the pharmaceutical industry. My PhD thesis work
led to multiple publications including a database of pediatric-specific
adverse drug effect signals aligning with dynamic physiological
processes during child development. I led the data science efforts to
develop an interpretable and robust machine learning algorithm that led
to hypothesized biological mechanism for a fatal, idiopathic graft
dysfunction after heart transplant surgery.

\begin{enumerate}
  \item \bibentry{peddb}
  \item \bibentry{pedreview}
  \item \bibentry{kks}
  \item \bibentry{klkb1}
\end{enumerate}

\section*{Positions and Honors}\label{positions-and-honors}
\addcontentsline{toc}{section}{Positions and Honors}

\subsection*{Positions and Employment}\label{positions-and-employment}
\addcontentsline{toc}{subsection}{Positions and Employment}

\begin{datetbl}
2023-- & Computer Science Advisor, MindArch Health \\
2021-- & Quantitative Transslational Scientist, Regeneron Pharmaceuticals \\
2016--2021 & Systems Biologist, Columbia University \\
2014--2019 & Cancer Bioinformatician, NHGRI \\
\end{datetbl}

\subsection*{Other Experience and Professional
Memberships}\label{other-experience-and-professional-memberships}
\addcontentsline{toc}{subsection}{Other Experience and Professional
Memberships}

\begin{datetbl}
2018--2018 & Bioinformatican, Genetic Leap \\
2021--2021 & Bioinformatician, DNAnexus \\
\end{datetbl}

\subsection*{Honors}\label{honors}
\addcontentsline{toc}{subsection}{Honors}

\begin{datetbl}
2018--2018 & Outstanding Contribution to Methodological Research at OHDSI symposium \\
2022--2022 & Travel Award to AMIA conference \\
\end{datetbl}

\section*{Contribution to Science}\label{contribution-to-science}
\addcontentsline{toc}{section}{Contribution to Science}

\begin{enumerate}

\item Side effects are significant safety concerns in pediatric drug treatment but are rarely captured during clinical trials and are severely underreported post-market. Moreover, variations in metabolism and physiology as children grow and develop complicate detection of drug safety signals across child development. We developed a novel machine learning approach for identifying ontogenic-mediated adverse event mechanisms. We then made a database of 500,000 drug safety signals, called KidSIDES, freely available and browsable by a web application.

\begin{enumerate}
  \item \bibentry{peddb}
  \item \bibentry{pedsim}
  \item \bibentry{pedreview}
\end{enumerate}


\item Primary graft dysfunction (PGD) is the leading cause of early mortality after heart transplant. Pre-transplant predictors of PGD remain elusive and its etiology remains unclear. A novel, patented machine learning algorithm identified pre-transplant level of KLKB1 is a robust predictor of post-transplant PGD. Our algorithm enabled the hypothesis that upregulation of coagulation cascade components of the kallikrein-kinin system (KKS) and downregulation of kininogen prior to transplant were associated with survival after transplant.

\begin{enumerate}
  \item \bibentry{kks}
  \item \bibentry{klkb1}
\end{enumerate}

\end{enumerate}



 

  \section{Research Support}
  
      
  
    \subsection*{Completed Research Support}
  
          
      \grantinfo{R35GM131905}{Nicholas Tatonetti}{2019-2024}
      {Data-driven drug discovery: investigating the molecular
mechanisms of safety and efficacy}
      {Prescription medicines are an essential component of modern
medicine, however, while these medicines work well for some patients,
they cause dangerous side effects in others. The lack of diversity in
clinical trials means that these effects may disproportionately affect
minorities and underrepresented patient populations. Using the
electronic health records, I propose to investigate the reasons for
adverse drug reactions in patients of different ages, sexes, genders,
and ancestries.}
      {Role: Investigator}
      
      \bigskip
      
  



\end{document}
